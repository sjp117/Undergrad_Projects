\documentclass[12pt]{article}
\setlength{\oddsidemargin}{0in}
\setlength{\evensidemargin}{0in}
\setlength{\textwidth}{6.5in}
\setlength{\topmargin}{-1in}
\setlength{\textheight}{9.7in}
\setlength{\parskip}{0.25\baselineskip}
\pagestyle{empty}

\usepackage{hyperref}
\usepackage{mathptmx}

\let\oldcenter\center
\let\oldendcenter\endcenter
\renewenvironment{center}{\setlength\topsep{-1pt}\oldcenter}{\oldendcenter}

\begin{document}
	
	\begin{center}
		{\Large Statement of Purpose} \\
		{\normalsize Sungjoon Park (\href{mailto:s99park@uwaterloo.ca}{s99park@uwaterloo.ca})}
	\end{center}
	
	I want to pursue a Ph.D to contribute to our understanding of the human mind and how it processes information and make decisions. Ultimately, I aim to become a professor. I prepared for graduate training through my research experience as a lab assistant, working on my thesis project, and by taking research oriented courses. Through my experience, I have come to appreciate concepts such as mental models and how they inform our decision making processes and the power of computational techniques in many aspects of research ranging from coding experiments to conducting statistical analysis. During my graduate training, I hope to continue this line of research while becoming more proficient in the use of computational tools and modeling techniques, and learn to use brain imaging tools to elucidate the neural substrate of learning and decision making.

	I prepared for graduate training with lab experience as an assistant and as an investigator for my own thesis project (in progress). In addition, I have taken research oriented courses that developed practical skills that involve psychophysical data collection, formulating survey questionnaires, coding a neural network in Python and performing data transformation, visualization and analysis with the R programming language. As a lab assistant, I assisted in two graduate student's projects, both involving eye-trackers and probability learning (PL). The first project investigated whether the manipulation of involuntary spatial attention can influence voluntary spatial attention. This was done by biasing participants to a region of a display with a spatial PL task and analyzing their voluntary attention tendency using the Tse illusion. The second project investigated what eye movements can reveal about mental model updating. It involved participants learning the distribution shape of how stimuli dots appeared on the surface of an invisible circle. We monitored eye behaviors, such as dwell duration (time spent fixated on a stimulus) and saccade latency (time between stimulus onset and saccade initiation) when stimuli appeared in `low' vs `high' probability locations and when the stimuli distribution shape was changed (`wide' $\leftrightarrow$ `narrow'). Through a directed studies \href{https://github.com/sjp117/Undergrad_Projects/tree/master/mixedEffectModelDiary}{project}\footnote{\url{https://github.com/sjp117/Undergrad\_Projects/tree/master/mixedEffectModelDiary}}, I developed my ability to work independently and I became more familiar with R programming through cleaning, mutating, visualizing and analyzing a diary data set. I learned to use packages to conduct quantitative discourse analysis, linear mixed effect modeling, and visualization.

	My thesis \href{https://github.com/sjp117/Undergrad_Projects/tree/master/mentalModelUpdatingPupil}{project}\footnote{\url{https://github.com/sjp117/Undergrad\_Projects/tree/master/mentalModelUpdatingPupil}} explores the relationships between changes in belief, confidence and pupil diameter. My participants were tasked to infer whether the shape or the color of the visual stimuli is relevant when making a decision to go `up' or `down', while they indicate how confident they feel that one of the factor is at play. After making their choice, they received a stochastic audio feedback where there was a small chance to be wrong regardless of making the correct choice. I manipulated the participants belief by alternating the relevant factor and I looked at their pupil responses when they made prediction errors. Of interest was comparing pupil response after experiencing an informative or an uninformative prediction error. A secondary hypothesis I explored was whether confidence positively correlated with greater pupil response and belief change. This was done by manipulating the stochasticity of the feedback where, during certain blocks, the chance of an unreliable feedback was increased.

	It would be my pleasure to work with the likes of Dr. Erie Boorman, Dr. Joy Geng, and Dr. Steve Luck, as their research is closely related to my interest. I look forward to conduct research in these topics while exploring different methods in the purview of reinforcement learning, and other machine learning model techniques.
	
\end{document}
