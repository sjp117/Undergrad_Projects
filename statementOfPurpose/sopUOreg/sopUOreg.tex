\documentclass[12pt]{article}
\setlength{\oddsidemargin}{0in}
\setlength{\evensidemargin}{0in}
\setlength{\textwidth}{6.5in}
\setlength{\topmargin}{-1in}
\setlength{\textheight}{9.7in}
\setlength{\parskip}{0.5\baselineskip}
\pagestyle{empty}

\usepackage{hyperref}
\usepackage{mathptmx}

\let\oldcenter\center
\let\oldendcenter\endcenter
\renewenvironment{center}{\setlength\topsep{-1pt}\oldcenter}{\oldendcenter}

\begin{document}
	
	\begin{center}
		{\Large Statement of Purpose} \\
		{\normalsize Sungjoon Park (\href{mailto:s99park@uwaterloo.ca}{s99park@uwaterloo.ca})}
	\end{center}
	
	I am most interested in how people construct internal models of their environment and how that model changes and how this process informs their decision making. Some of the questions I wish to ask are in regards to how we can construct a computational model of these processes. How do prediction errors inform the state of the environment? What are the neural substrate of internal models? What role does the dopamine and the norepinephrine system play? My research experience have prepared me to answer these questions through assisting in projects that investigated mental model updating and by familiarizing me with computational tools that allow me to construct behavioral experimental tasks and conduct data analysis. I believe this institution is an excellent institution to continue my research and training because of the faculty members who share my interest and the wealth of facilities I can utilize to answer my questions. My goals after my graduate training is to continue a career path to become a professor who is well equipped to take their research anywhere, be it in North America or back in my native country, South Korea, to contribute to a global initiative to practice cognitive science and to foster future scholars.
	
	While my interest in the human mind started during my childhood, my research experience only started during my third undergraduate year at the University of Waterloo. Before that, I was on a two years leave to fulfill the South Korean military service requirement. I used this period as an opportunity to consider many career options (both academic and non-academic) and I invested time to read on topics ranging from military psychology, history, computer science, philosophy, behavioral economics, biology, and neuroscience. Ultimately, I decided to return to my undergraduate study with a galvanized will to continue a career path to become a scholar in the field of cognitive science. After my return, I entered Dr. Britt Anderson’s lab as a research assistant, and I enrolled in a directed studies course on statistics, supervised by Dr. Anna Dorfman.
	
	In the lab, I assisted in two graduate student's projects, both involving eye-trackers (SR Eyelink) and probability learning (PL). The first project investigated whether the manipulation of involuntary spatial attention can influence voluntary spatial attention. This was done by biasing participants to a region of a display with a spatial PL task and analyzing their voluntary attention tendency using the Tse illusion. The second project investigated what eye movements can reveal about mental model updating. It involved participants learning the distribution shape of how stimuli dots appeared on the surface of an invisible circle. We monitored eye behaviors such as dwell duration (time spent fixated on a stimulus) and saccade latency (time between stimulus onset and saccade initiation) when stimuli appeared in `low' vs `high' probability locations and when the stimuli distribution shape was changed (`wide' $\leftrightarrow$ `narrow').
	
	Through a directed studies course, I strengthened my ability to work independently and I became more familiar with R programming through cleaning, mutating, visualizing and analyzing a diary data set. This data set was originally used to explore how people internalize events when writing a diary in either a self-immersed (first-person) or self-distanced (third-person) perspective. I used this data set to conduct my own exploratory analysis with my own post-hoc hypotheses. I learned to use packages to conduct quantitative discourse analysis, linear mixed effect modeling, and visualization. Along side my final write up, I created a supplementary document walking through my analysis and visualization process; which is openly  \href{https://github.com/sjp117/Undergrad_Projects/tree/master/mixedEffectModelDiary}{accessible}\footnote{\url{https://github.com/sjp117/Undergrad\_Projects/tree/master/mixedEffectModelDiary}}.
	
	Currently, I am working on my undergraduate thesis \href{https://github.com/sjp117/Undergrad_Projects/tree/master/mentalModelUpdatingPupil}{project}\footnote{\url{https://github.com/sjp117/Undergrad\_Projects/tree/master/mentalModelUpdatingPupil}} under the supervision of Dr. Britt Anderson. This project explores the relationships between changes in an internal model, confidence and pupil diameter. My participants were tasked to infer whether the shape or the color of the visual stimuli was relevant when making a decision to go `up' or `down', and indicate how confident they feel that one or the other factor is at play. After making their choice, they received a stochastic audio feedback where there was a small chance to be wrong regardless of making the correct choice. I manipulated the participants belief by alternating the relevant factor while I looked at their pupil responses when they made prediction errors. Of interest was comparing pupil response after experiencing an informative or an uninformative prediction error. I hypothesized that informative errors will elicit a greater pupil response. A secondary hypothesis I explored was whether confidence positively correlated with greater pupil response and belief change. This was done by manipulating the stochasticity of the feedback where, during certain blocks, the chance of an unreliable feedback was increased.
	
	In the process of working on my thesis project, I developed a variety of technical skills. I became more skilled in Python and the use of the Psychopy library to code my experiment. I became more proficient with R programming to transform, visualize and analyze data. I applied parallelization to some of my Python codes to run concurrently with another task or to expedite a process. I learned to use an unfamiliar eye tracker (CRS LiveTrack), where I needed to familiarize with its code library and troubleshooting its bugs. And, I became more proficient working in the Unix environment and a variety of its tools to troubleshoot hardware issues and maintain a backup pipeline. I believe the theoretical and technical skills I developed will be of value for some of the faculty members who share my research interest.
	
	Among faculty members, it would be my pleasure to work with the likes of Dr. Benjamin Hutchinson, and Dr. Sarah DuBrow. My interest most closely align with Dr.  Hutchinson’s as I am interested in how our past experience construct internal models that allow us to quickly decide what sensory information needs to be attended to. For instance, when crossing a road, we are able to rapidly drown out distractions such as bright neon signs and loud street performers while we pay attention to the motion of fellow pedestrians, the traffic and the pedestrian lights. How are these rules learned? How do they change (update)? What are their neural substrates? And could we model them in the purview of reinforcement learning or other machine learning techniques with realistic energetic constraints?
	
	Dr. DuBrow’s work is of interest in the context of event boundaries. The brain seems to be able to accumulate evidence where when it reaches a threshold, it signals a change in event that suggests the need to reevaluate the current environmental state and update an internal model. Again in the context of crossing a road, a variety of signals such as the individual’s goal of reaching a destination across the road, the spatial sign that one is reaching the end of the pedestrian walk, and the diminishing number of pedestrian, all signal a need to attend to the traffic and the pedestrian light instead of the friend they are chatting with. What are the neural substrate of this processes? Once the current state has been decided, what acts as the filter that allows us to attend to the relevant features? How could we computationally model this behavior? I am eager to learn new computational and brain imaging methods to answer these questions. While my interest is fairly solidified, I am open to new ideas and I would gladly explore different aspects of cognition, using different research methodologies.
	
\end{document}
