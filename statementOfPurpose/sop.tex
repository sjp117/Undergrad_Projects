\documentstyle[12pt]{article}
\setlength{\oddsidemargin}{0in}
\setlength{\evensidemargin}{0in}
\setlength{\textwidth}{6.5in}
\setlength{\topmargin}{-.3in}
\setlength{\textheight}{9in}
\pagestyle{empty}

\begin{document}

\begin{center}
{\Large Statement of Purpose} \\[.1in]
{\large Sungjoon Park}
\end{center}

\vspace*{.2in}

I want to pursue a Ph.D to contribute to our understanding of the human mind and how it processes information. I am interested in human cognition and how it uses mental models to maintain a snap shot of the environment in order to make decisions. I am also interested in using computational techniques in my research ranging from coding experiments, to performing data cleaning, visualization and statistical analysis. In the future, I wish to expand my skill set to include computational modeling techniques that can involve Bayesian and machine learning elements.

Currently, I am working on my undergraduate thesis under the advisory of Dr. Britt Anderson. My project explores the potential relationship between mental model updating (changes in expectations or belief) and pupil diameter. My participants were tasked to decide whether the shape or the color of the visual stimuli were relevant when making a decision to go “up” or “down”. I manipulated the participants mental model by alternating between which factor was relevant and I looked at their pupil responses when they made prediction errors. After experiencing a prediction error, participants can suspect something has changed and decide to update their mental model. However, not all prediction errors are informative. In my case, when a prediction error involved both color and shape pointing at the same direction, it was uninformative. Alternatively, when the color and the shape pointed at different directions, it was informative. A secondary hypothesis I explored was whether confidence positively correlated with the degree of mental model updating; where prediction errors during high confidence could be more salient and lead to a greater pupil response.

In the process of working on my thesis project, I have demonstrated my ability to work and learn independently. I was able to code my experiment with Python and the Psychopy library with limited assistance. I also implemented parallelization into my code where the data storage was performed concurrently on a separate computing thread along side the main experimental protocol. I developed a script which converted the binary data output from my experiment into a data frame CSV file using multiprocessing which could allocate multiple computing process to expedite the conversion. For the choice of the eye-tracker, I took the initiative to work with a new lab hardware (CRS LiveTrack) with basic documentation and no experienced personnel available. I was able to familiarize with its code library and uncovered bugs and found solutions to it. When analyzing the pupil data, I coded my own analysis script in R which included preprocessing steps such as blink detection/extension, data interpolation and noise filtering. In addition, I utilized a high refresh rate monitor which required modification to a computer’s display setting through a command line interface. Throughout the process, I became more familiar with using web searches to find solutions to problems, using the Unix environment, and tools such as version control (git), file transferring/synchronization (rsync), and scheduling (cron). As a proponent for open science, I have my project publicly available in a github repository and I make the effort to share my project materials such as my proposal draft, ethics application, and recruitment materials with anyone in need.

My research experience started during my third undergraduate year in the University of Waterloo. Before that, I was on a two years leave to fulfill the South Korean military service requirement. After my return, I entered Dr. Britt Anderson’s lab as a research assistant and enrolled in a directed studies course on statistics, supervised by Dr. Anna Dorfman. My time in the S.Korean military was spent evaluating my career path. I had the opportunity to consider many options and conduct research into many fields. I invested time to read on topics ranging from military psychology, history, computer science, philosophy, behavioral economics, biology, neuroscience, and cognitive psychology. Ultimately, I decided to return to my undergraduate study with a galvanized will to continue a career path to become researcher and an academic in the field of cognitive psychology.

In the lab, I assisted in two graduate student’s projects, both involving eye-trackers (SR Eyelink) and probability learning (PL). The first project investigated whether the manipulation of involuntary spatial attention can influence voluntary spatial attention. This was accomplished by comparing the spatial attention tendency between one group of participants who were biased to a region of space through a spatial PL task and another group that was not. Testing whether this manipulation influenced voluntary spatial attention was accomplished by presenting the Tse illusion before and after the PL task. The Tse illusion is characterized by three, partially overlapping, gray circles with a central fixation point. It is known for how the circle the observer is voluntarily attending to is reported to be darker. I was chiefly involved in the data collection process where I received participants and instructed them how to perform the task. I also operated the eye-tracker for calibration and recording. Ultimately, while the probability learning task did manipulate involuntary spatial attention, the study could not demonstrate that involuntary spatial attention can manipulate voluntary spatial attention.

The second project investigated what eye movements can reveal about mental model updating. This experiment involved participants learning a distribution pattern of how dot stimuli appeared on the perimeter of an invisible circle. At one point, the shape of the distribution changed from wide to narrow (or vice versa). When the distribution changed, participants were expected to be surprised by stimuli appearing in unlikely locations, motivating a mental model update and changing the shape of the distribution they believe the stimuli will follow. An eye tracker was used to monitor looking behaviors such as saccade latency, and dwell duration. I was involved in the majority of the data collection process where I received participants, guiding them through the experiment and in the operation the eye tracker. I was also involved in the data analysis process, where I assisted in the formulation and the coding of the analysis and the interpretation of it. We found that, dwell duration (time spent fixated on a stimulus) was positively correlated with surprise, concluding that it was sensitive to mental model updating, and that saccade latency (time between stimulus onset and saccade initialization) was faster for high probability regions on learned distributions. (what I learned)

During the directed studies course in statistics, I attempted to become more familiar with the R programming language and conducting linear mixed effect modeling. I was provided a diary study data set that investigated how people might internalize their experience when instructed to write their diary in either a first-person (self-immersed) or a third-person (self-distanced) perspective. I conducted an exploratory analysis with my own post-hoc hypotheses. During this course, the majority of the work was done independently and I became more familiar with R programming through cleaning, mutating, visualizing and analyzing the data set. I learned to use packages to conduct quantitative discourse analysis (qdap), linear mixed effect modeling (lme4), and visualization (ggplot2). In addition, I became more familiar with the tidyverse, data.frame, and data.table syntaxes. While my results were negative, I was able to develop valuable skills in working independently. Along side my final write up, I created a supplementary document, walking through my analysis and visualization process which is openly available in a github repository.

Although I am chiefly interested in how people process information and make decisions, I am open to a verity of topics about human cognition. I wish to continue my study and research at...

\end{document}
