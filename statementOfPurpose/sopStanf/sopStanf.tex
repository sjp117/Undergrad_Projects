\documentclass[12pt]{article}
\setlength{\oddsidemargin}{0in}
\setlength{\evensidemargin}{0in}
\setlength{\textwidth}{6.5in}
\setlength{\topmargin}{-1in}
\setlength{\textheight}{9.7in}
\setlength{\parskip}{0.5\baselineskip}
\pagestyle{empty}

\usepackage{hyperref}
\usepackage{mathptmx}

\let\oldcenter\center
\let\oldendcenter\endcenter
\renewenvironment{center}{\setlength\topsep{-1pt}\oldcenter}{\oldendcenter}

\begin{document}
	
	\begin{center}
		{\Large Statement of Purpose} \\
		{\normalsize Sungjoon Park (\href{mailto:s99park@uwaterloo.ca}{s99park@uwaterloo.ca})}
	\end{center}
	
	I am interested in how people construct mental models and how it informs their decision making process. I also wish to explore how we might construct computational models of this process. My research experience has prepared me to ask these questions through assisting in and conducting my own research in mental model updating and by familiarizing me with computational tools that allow me to construct behavioral experimental tasks, collect psychophysical data and conduct data analysis. I believe this institution is ideal for continuing my training and research because of the faculty members who share my interest and its wealth of material resources. My goal after graduate training is to continue a career path to become a professor who is well equipped to take their research anywhere, be it in North America or back in my native country, South Korea, to contribute to a global initiative to practice cognitive science and to foster future scholars.
	
	While my interest in the human mind started during my childhood, my research experience only started during my third year at the University of Waterloo. Before that, I was on a two years leave to fulfill the South Korean military service requirement. I used this period as an opportunity to consider many career options (both academic and non-academic) and I invested time to read on a variety of topics. By the end of my service, I developed greater appreciation for philosophy (epistemology, morality) and computer science. Ultimately, I decided to return to my undergraduate study with a galvanized will to continue a career path to become a scholar in the field of cognitive science. When I returned, I entered Dr. Britt Anderson’s lab as a research assistant, and I enrolled in a directed studies course on statistics, supervised by Dr. Anna Dorfman, to prepare for my future endeavor.
	
	In the lab, I assisted in two graduate student's projects, both involving eye-trackers (SR Eyelink) and probability learning (PL). The first project investigated whether the manipulation of involuntary spatial attention can influence voluntary spatial attention. This was done by biasing participants to a region of a display with a spatial PL task and analyzing their voluntary attention tendency using the Tse illusion. The second project investigated what eye movements can reveal about mental model updating. It involved participants learning the distribution shape of how stimuli dots appeared on the surface of an invisible circle. We monitored eye behaviors such as dwell duration (time spent fixated on a stimulus) and saccade latency (time between stimulus onset and saccade initiation) when stimuli appeared in `low' vs `high' probability locations and when the stimuli distribution shape was changed (`wide' $\leftrightarrow$ `narrow').
	
	Through a directed studies course, I strengthened my ability to work independently and I became more familiar with R programming through cleaning, mutating, visualizing and analyzing a diary data set. This data set was originally used to explore how people internalize events when writing a diary in either a self-immersed (first-person) or self-distanced (third-person) perspective. I used this data set to conduct my own exploratory analysis with my own post-hoc hypotheses. I learned to use packages to conduct quantitative discourse analysis, linear mixed effect modeling, and visualization. Along side my final write up, I created a supplementary document walking through my analysis and visualization process; which is openly  \href{https://github.com/sjp117/Undergrad_Projects/tree/master/mixedEffectModelDiary}{accessible}\footnote{\url{https://github.com/sjp117/Undergrad\_Projects/tree/master/mixedEffectModelDiary}}.
	
	Currently, I am working on my undergraduate thesis \href{https://github.com/sjp117/Undergrad_Projects/tree/master/mentalModelUpdatingPupil}{project}\footnote{\url{https://github.com/sjp117/Undergrad\_Projects/tree/master/mentalModelUpdatingPupil}} under the supervision of Dr. Britt Anderson. This project explores the relationships between changes in an mental model, confidence and pupil diameter. My participants were tasked to infer whether the shape or the color of the visual stimuli was relevant when making a decision to go `up' or `down', and indicate how confident they feel that one or the other factor is at play. After making their choice, they received a stochastic audio feedback where there was a small chance to be wrong regardless of making the correct choice. I manipulated the participants belief by alternating the relevant factor while I looked at their pupil responses when they made prediction errors. Of interest was comparing pupil response after experiencing an informative or an uninformative prediction error. I hypothesized that informative errors will elicit a greater pupil response. A secondary hypothesis I explored was whether confidence positively correlated with greater pupil response and belief change. This was done by manipulating the stochasticity of the feedback where, during certain blocks, the chance of an unreliable feedback was increased.
	
	In the process of working on my thesis project, I developed a variety of technical skills. I became more skilled in the use of Python and the Psychopy library to code my experiment. I became more proficient with R programming to transform, visualize and analyze data. I applied parallelization to some of my Python codes to run tasks concurrently or to expedite a process. I learned to use an unfamiliar eye tracker (CRS LiveTrack), applying its code library and resolving bugs. And, I became more proficient working in the Unix environment and a variety of its tools to troubleshoot hardware issues and maintain a backup pipeline. I believe the skills I developed will be of value to both research and teaching assistantship. Although I’m most familiar with Python and R, I am confident I can efficiently adapt to different programming languages, such as MATLAB or Julia, if there is motivation to do so.
	
	I am most interested to work with Dr. Tobias Gerstenberg as my interest is most aligned with his. I want to investigate how our mind is able to construct and use mental models to inform our decision making process. Of interest is the mechanism behind the model selection process. Out of all the possible models of the world, our mind seems to efficiently select the appropriate one (in most cases). For example, while walking down a sidewalk and chatting with a friend, we adopt a mental model that is relevant to the conversation and we are sensitive inconsistencies in the narrative. However, when we realize we are coming to the end of the side walk, we rapidly switch our mental model to attend to the motion of the traffic and the state of the pedestrian light, while ignoring other traits, such as the color or the brand of the vehicles in the traffic. With little deliberation, our mind was able to infer which causal elements to consider when deciding to cross the road.
	
	After making a decision, we lose utility of the model and we seamlessly transition back to the model pertinent to the friendly conversation. Perhaps our value of the conversation kept its model at the top of a queue (how does the mind know what to value?). Maybe the mind decided that the most recent, unresolved, model should be of top priority. Whatever the mechanism may be, I also wish to investigate where this process happens, where the models are stored and what type of error signals accumulate when deciding which model to implement (dopamine system in the striate? Norepinephrine system such as the LC? OFC for more deliberate model selection?). Furthermore, how might we construct a computational model of this process? (In the purview of reinforcement learning or other machine learning methods.)
	
	My questions are numerous and one could become an expert in answering just one of them. As such, I am willing to narrow down my research program and I am open to considering other research questions if suggested by any potential mentor.
	
\end{document}
