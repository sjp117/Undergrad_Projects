\documentclass[12pt]{article}
\setlength{\oddsidemargin}{0in}
\setlength{\evensidemargin}{0in}
\setlength{\textwidth}{6.5in}
\setlength{\topmargin}{-1in}
\setlength{\textheight}{9.9in}
\setlength{\parskip}{0.5\baselineskip}
\pagestyle{empty}

\usepackage{hyperref}
\usepackage{mathptmx}

\let\oldcenter\center
\let\oldendcenter\endcenter
\renewenvironment{center}{\setlength\topsep{-1pt}\oldcenter}{\oldendcenter}

\begin{document}

\begin{center}
{\Large Personal Statement} \\
{\normalsize Sungjoon Park (\href{mailto:s99park@uwaterloo.ca}{s99park@uwaterloo.ca})}
\end{center}

I want to pursue a Ph.D. to contribute to our understanding of the human mind and how it processes information to make decisions. Through my undergraduate research, I have come to appreciate the importance of mental models and how it relates to the decision making processes and the power of computational techniques in many aspects of research ranging from coding experiments to conducting statistical analysis. During my graduate training, I hope to continue my line of research while becoming more proficient with computational tools, that can involve Bayesian and machine learning elements, and in the use of brain imaging tools. As such, I believe continuing my training in this institution with its faculty members is invaluable to fulfilling my career goal of becoming a professor.

It would be my pleasure to work with faculty members such as Dr. Joe Kable and Dr. Joshua Gold (Neuroscience) to explore the neural mechanisms involved in belief changes and attempt to construct computational models of how different types of surprises update our beliefs of various kinds (statistical, rational, emotional, and etc.) and complexity. I am also interested to learn and apply functional imaging and stimulation techniques to explore how the norepinephrine and dopamine systems relate to belief updating. My relative familiarity with the belief updating literature and experience with eye trackers for pupil studies may be of asset to continue work similar to that of \href{https://doi.org/10.1101/736140 }{Filipowicz, Glaze, Kable, \& Gold (2019)}\footnote{\url{https://doi.org/10.1101/736140}}.

Other faculty members of interest are Dr. Anna Schapiro and Dr. Sudeep Bhatia. Under Dr. Schapiro, I envision exploring the relationship between short-term and long-term belief changes and how the hippocampus and sleep are associated with it. In addition, Dr. Schapiro's familiarity with neural-network modeling and brain imaging tools are of great interest to me. Working with Dr. Bhatia would be a great opportunity to learn and apply a variety of computational modeling tools. I could explore belief updating in the purview of evidence accumulation models and compare how they differ for different types of evidence and beliefs. Although I am interested in belief updating, I am not fixated on it. I am willing to take on a variety of projects, and I am willing to modify my research interest if suggested by a potential mentor.

Although my interest in the human mind persisted since my childhood, my research experience only started during my third year at the University of Waterloo. Before that, I was on a two years leave to fulfill the South Korean military service requirement. I used this period as an opportunity to consider many career options (both academic and non-academic) and I invested time to read on topics ranging from military psychology, history, computer science, philosophy, behavioral economics, biology, and neuroscience. Ultimately, I decided to return to my undergraduate study with a galvanized will to continue a career path to become a scholar in the field of cognitive science. When I returned, I entered Dr. Britt Anderson’s lab as a research assistant, and I enrolled in a directed studies course on statistics, supervised by Dr. Anna Dorfman, to prepare for my future endeavors.

Currently, I am working on my undergraduate thesis \href{https://github.com/sjp117/Undergrad_Projects/tree/master/mentalModelUpdatingPupil}{project}\footnote{\url{https://github.com/sjp117/Undergrad\_Projects/tree/master/mentalModelUpdatingPupil}} under the supervision of Dr. Britt Anderson. This project explores the relationships between changes in an internal model, confidence and pupil diameter. My participants were tasked to infer whether the shape or the color of the visual stimuli was relevant when making a decision to go `up' or `down', and indicate how confident they feel that one or the other factor is at play. After making their choice, they received a stochastic audio feedback where there was a small chance to be wrong regardless of making the correct choice. I manipulated the participants belief by alternating the relevant factor while I looked at their pupil responses when they made prediction errors. Of interest was comparing pupil response after experiencing an informative or an uninformative prediction error. I hypothesized that informative errors will elicit a greater pupil response. A secondary hypothesis I explored was whether confidence positively correlated with greater pupil response and belief change. This was done by manipulating the stochasticity of the feedback where, during certain blocks, the chance of an unreliable feedback was increased.

In the process of working on my thesis project, I developed a variety of technical skills. I became more skilled in the use of Python and the Psychopy library to code my experiment. I became more proficient with R programming to transform, visualize and analyze data. I applied parallelization to some of my Python codes to run tasks concurrently or to expedite a task. I learned to use an unfamiliar eye tracker (CRS LiveTrack), applying its code library and resolving bugs. And, I became more proficient working in the Unix environment and a variety of its tools to troubleshoot hardware issues and maintain a backup pipeline. I believe the skills I developed will be of value to both research and teaching assistantship. Although I’m most familiar with Python and R, I am confident I can efficiently adapt to different programming languages, such as MATLAB.

In the lab, I assisted in two graduate student's projects, both involving eye-trackers (SR Eyelink) and probability learning (PL). The first project investigated whether the manipulation of involuntary spatial attention can influence voluntary spatial attention. This was done by biasing participants to a region of a display with a spatial PL task and analyzing their voluntary attention tendency using the Tse illusion. The second project investigated what eye movements can reveal about mental model updating. It involved participants learning the distribution shape of how stimuli dots appeared on the surface of an invisible circle. We monitored eye behaviors such as dwell duration (time spent fixated on a stimulus) and saccade latency (time between stimulus onset and saccade initiation) when stimuli appeared in `low' vs `high' probability locations and when the stimuli distribution shape was changed (`wide' $\leftrightarrow$ `narrow').

Through a directed studies course, I strengthened my ability to work independently and I became more familiar with R programming through cleaning, mutating, visualizing and analyzing a diary data set. This data set was originally used to explore how people internalize events when writing a diary in either a self-immersed (first-person) or self-distanced (third-person) perspective. I used this data set to conduct my own exploratory analysis with my own post-hoc hypotheses. I learned to use packages to conduct quantitative discourse analysis, linear mixed effect modeling, and visualization. Alongside my final write up, I created a supplementary document walking through my analysis and visualization process; which is openly  \href{https://github.com/sjp117/Undergrad_Projects/tree/master/mixedEffectModelDiary}{accessible}\footnote{\url{https://github.com/sjp117/Undergrad\_Projects/tree/master/mixedEffectModelDiary}}.

\end{document}
