\documentclass[12pt]{article}
\setlength{\oddsidemargin}{0in}
\setlength{\evensidemargin}{0in}
\setlength{\textwidth}{6.5in}
\setlength{\topmargin}{-1in}
\setlength{\textheight}{9.7in}
\setlength{\parskip}{0.5\baselineskip}
\pagestyle{empty}

\usepackage{hyperref}
\usepackage{mathptmx}

\let\oldcenter\center
\let\oldendcenter\endcenter
\renewenvironment{center}{\setlength\topsep{-1pt}\oldcenter}{\oldendcenter}

\begin{document}
	
	\begin{center}
		{\Large Statement of Purpose} \\
		{\normalsize Sungjoon Park (\href{mailto:s99park@uwaterloo.ca}{s99park@uwaterloo.ca})}
	\end{center}
	
	I want to pursue a Ph.D to contribute to our understanding of the human mind and how it processes information and make decisions. Ultimately, I aim to become a professor. Through my undergraduate research, I have come to appreciate concepts such as mental models and how it relates to our decision making processes and the power of computational techniques in many aspects of research ranging from coding experiments to conducting statistical analysis. During my graduate training, I hope to continue my research in human information processing and decision making, while becoming proficient in the use of computational tools and modeling techniques (that can involve Bayesian and machine learning elements), and brain imaging tools to investigate the neural substrates of these functions of interest.
	
	Currently, I am working on my undergraduate thesis \href{https://github.com/sjp117/Undergrad_Projects/tree/master/mentalModelUpdatingPupil}{project}\footnote{\url{https://github.com/sjp117/Undergrad\_Projects/tree/master/mentalModelUpdatingPupil}} under the supervision of Dr. Britt Anderson. This project explores the relationships between changes in belief, confidence and pupil diameter. My participants were tasked to infer whether the shape or the color of the visual stimuli was relevant when making a decision to go `up' or `down', while they indicate how confident they feel that one of the factor is at play. After making their choice, they received a stochastic audio feedback where there was a small chance to be wrong regardless of making the correct choice. I manipulated the participants belief by alternating the relevant factor and I looked at their pupil responses when they made prediction errors. Of interest was comparing pupil response after experiencing an informative or an uninformative prediction error. A secondary hypothesis I explored was whether confidence positively correlated with greater pupil response and belief change. This was done by manipulating the stochasticity of the feedback where, during certain blocks, the chance of an unreliable feedback was increased.
	
	In the process of working on my thesis project, I developed a variety of technical skills. I learned to code in Python and use the Psychopy library to code my experiment. I became more proficient with R programming to transform, visualize and analyze data. I applied parallelization to some of my Python codes to run concurrently with another task or to expedite a process. I learned to use an unfamiliar eye tracker (CRS LiveTrack); familiarizing with its code library and troubleshooting its bugs. And, I became more proficient working in the Unix environment and a variety of its tools. 
	
	My research experience started during my third undergraduate year at the University of Waterloo. Before that, I was on a two years leave to fulfill the South Korean military service requirement. I used this period as an opportunity to consider my career options (both academic and non-academic) and I invested time to read about many subjects. Ultimately, I decided to return to my undergraduate study with a galvanized will to continue my initial desire to become scholar in cognitive science. After my return, I entered Dr. Britt Anderson's lab as a research assistant and enrolled in a directed studies course on statistics, supervised by Dr. Anna Dorfman.
	
	In the lab, I assisted in two graduate student's projects, both involving eye-trackers (SR Eyelink) and probability learning (PL). The first project investigated whether the manipulation of involuntary spatial attention can influence voluntary spatial attention. This was done by biasing participants to a region of a display with a spatial PL task and analyzing their voluntary attention tendency using the Tse illusion. The second project investigated what eye movements can reveal about mental model updating. It involved participants learning the distribution shape of how stimuli dots appeared on the surface of an invisible circle. We monitored eye behaviors, such as dwell duration (time spent fixated on a stimulus) and saccade latency (time between stimulus onset and saccade initiation) when stimuli appeared in `low' vs `high' probability locations and when the stimuli distribution shape was changed (`wide' $\leftrightarrow$ `narrow'). Through a directed studies course, I developed my ability to work independently and I became more familiar with R programming through cleaning, mutating, visualizing and analyzing a diary data set. I learned to use packages to conduct quantitative discourse analysis, linear mixed effect modeling, and visualization. Along side my final write up, I created a supplementary document walking through my analysis and visualization process which is openly available in a github \href{https://github.com/sjp117/Undergrad_Projects/tree/master/mixedEffectModelDiary}{repository}\footnote{\url{https://github.com/sjp117/Undergrad\_Projects/tree/master/mixedEffectModelDiary}}.
	
	I am most interested in working with Dr. Jason Samaha as his research is most similar to my interest. I would like to explore what role confidence plays in the process of evidence accumulation. Do people tend to give contrary evidence greater weight when they are less confident about the current state of the environment? Or is it the other way around? How could we model this in the purview of reinforcement learning? Does confidence relate to the learning rate or the error signal of our observations? How does the confidence in one’s ability and confidence in the stability of the environment relate to decision making? In addition, I am also interested in how the mind decides to drown out some stimuli while attend to others. How is it that, when crossing a road, people quickly focus on the motion of the surrounding traffic or the traffic lights instead of the bright neon signs or the loud street performers? Unlike machine learners, the human brain is only privy to a limited amount of energy. Yet it is able to make quick and complex decisions. I wish to understand how this is possible and even contribute to the field of artificial intelligence to devise a more human-like and efficient artificial agent.
	
	Other faculty members I could work with are Dr. Benjamin C. Storm and Dr. Nicolas Davidenko. With Dr. Storm, I can investigate the long term memory, effect of learning. What role does confidence play in the formation of memory? What is the relationship between the short term change in belief and its longevity? What features does the mind decide to encode in long term? Is there a process of simplification involved in long term rule learning? How might this long term reinforcement learning model differ from a typical short term version? With Dr. Davidenko, I’m intrigued by the notion of the suggestibility of perception. It would be interesting to investigate the priming tendency of our senses. Model how certain stimuli prime our senses and how it reverts back to baseline. However, I am not married to my interest in learning and decision making. I am open to investigating different research questions that we devise.
	
\end{document}
